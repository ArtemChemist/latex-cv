%%%%%%%%%%%%%%%%%%%%%%%%%%%%%%%%%%%%%%%%%
% Developer CV
% LaTeX Template
% Version 1.0 (28/1/19)
%
% Based on a template by Jan Vorisek (jan@vorisek.me)
% https://www.latextemplates.com/template/developer-cv
%
% License:
% The MIT License (see included LICENSE file)
%
%%%%%%%%%%%%%%%%%%%%%%%%%%%%%%%%%%%%%%%%%

%----------------------------------------------------------------------------------------
%	PACKAGES AND OTHER DOCUMENT CONFIGURATIONS
%----------------------------------------------------------------------------------------

\documentclass[9pt]{developercv} % Default font size, values from 8-12pt are recommended

%----------------------------------------------------------------------------------------

\begin{document}

%----------------------------------------------------------------------------------------
%	TITLE AND CONTACT INFORMATION
%----------------------------------------------------------------------------------------

\begin{minipage}[t]{0.3\textwidth} % 45% of the page width for name
	\vspace{-\baselineskip} % Required for vertically aligning minipages
	
	% If your name is very short, use just one of the lines below
	% If your name is very long, reduce the font size or make the minipage wider and reduce the others proportionately
	\colorbox{black}{{\huge\textcolor{white}{\textbf{\MakeUppercase{Artem Lebedev}}}}} % First name
	
	\vspace{6pt}
	
	{\huge MS in Data Science}\\ % Career or current job title
	
	{\huge PhD in Life Science} % Career or current job title
\end{minipage}
\begin{minipage}[t]{0.35\textwidth} % 27.5% of the page width for the first row of icons
	\vspace{-\baselineskip} % Required for vertically aligning minipages
	
	% The first parameter is the FontAwesome icon name, the second is the box size and the third is the text
	% Other icons can be found by referring to fontawesome.pdf (supplied with the template) and using the word after \fa in the command for the icon you want
	\icon{MapMarker}{12}{Hamilton, ON Canada}\\
	\icon{Phone}{12}{+1 365 888 8815}\\
	\icon{At}{12}{\href{mailto:artem@indiechemistry.com}{artem@indiechemistry.com}}\\	
\end{minipage}
\begin{minipage}[t]{0.35\textwidth} % 27.5% of the page width for the second row of icons
	\vspace{-\baselineskip} % Required for vertically aligning minipages
	
	% The first parameter is the FontAwesome icon name, the second is the box size and the third is the text
	% Other icons can be found by referring to fontawesome.pdf (supplied with the template) and using the word after \fa in the command for the icon you want
	\icon{Linkedin}{12}{\href{https://www.linkedin.com/in/artemlebedev/}{linkedin.com/in/artemlebedev}}\\
	\icon{Github}{12}{\href{https://github.com/ArtemChemist}{github.com/ArtemChemist}}\\
	\icon{GraduationCap}{12}{\href{https://scholar.google.com/citations?hl=en&user=ybQleeAAAAAJ}{Google Scholar}}\\
\end{minipage}

\vspace{0.5cm}

%----------------------------------------------------------------------------------------
%	INTRODUCTION, SKILLS AND TECHNOLOGIES
%----------------------------------------------------------------------------------------

\begin{minipage}[t]{0.35\textwidth} % 40% of the page width for the introduction text
	\vspace{-\baselineskip} % Required for vertically aligning minipages
	\cvsect{Data Science} % Skill subsection
	
	TensorFlow $\cdot$ Keras $\cdot$ Scikit-learn\\
	PySpark $\cdot$ Hadoop \\ MLlib
	R  $\cdot$ tydyverse \\
	Anova $\cdot$ bootstrap $\cdot$ hypothesis testing\\
	CNN$\cdot$Random Forest$\cdot$ GLM $\cdot$ Time Series  $\cdot$\\
\end{minipage}
\hspace{0.005\textwidth}
\begin{minipage}[t]{0.31\textwidth} % 40% of the page width for the introduction text
	\vspace{-\baselineskip} % Required for vertically aligning minipages
	\cvsect{Dev and DevOps} % Skill subsection
	
	Python $\cdot$ Postgres SQL $\cdot$ MongoDB \\
	Linux $\cdot$ Amazon EC2 $\cdot$ GCloud\\
	Databricks $\cdot$ Flask $\cdot$ Gunicorn $\cdot$ NGNX \\
	VB.NET $\cdot$ WinForms\\
	Git CLI $\cdot$ Conda CLI $\cdot$ \LaTeX
\end{minipage}
\hspace{0.005\textwidth}
\begin{minipage}[t]{0.33\textwidth} % 40% of the page width for the introduction text
	\vspace{-\baselineskip} % Required for vertically aligning minipages
	\cvsect{Life Science} % Skill subsection
	
	Oncology $\cdot$ Hypoxia $\cdot$ Inflammation\\
	Molecular Imaging $\cdot$ Pharmacology\\
	Small animals $\cdot$ Cell cultures\\
	Wet chemistry $\cdot$ Radiolabeling\\
	Clinical trials $\cdot$ Method validation\\
\end{minipage}

%----------------------------------------------------------------------------------------
%	PROJECTS
%----------------------------------------------------------------------------------------

\cvsect{Projects}

\begin{entrylist}
	\entry
	{2021 -- 2023}
	{Detection of bacterial colonies in Petri Dishes}
	{Feasibility study for LabLogic Inc.}
	{Image classification to "sterile" and "contaninated" Petri Dishes, based on pre-trained neural networks: Efficient Net, VGG16, ResNet, Inception \\ \texttt{OpenCV}\slashsep\texttt{CNN}\slashsep\texttt{YOLOv8}\slashsep\texttt{Python}}
	\entry
	{2023 -- \\\footnotesize{present}}
	{Sales analysis and forecasting}
	{McMaster Nuclear Reactor}
	{Created a sales analysis web portal: a PostGreSQL database coupled to a Flask-based web app, running on Gunicorn webserver. The data allows for SARIMA time-series based sales forecasts.\\ \texttt{Flask}\slashsep\texttt{SQL}\slashsep\texttt{Time series}\slashsep\texttt{PostGreSQL}}
	\entry
	{2013 -- 2019}
	{Optical spectra analysis for Quality Control System}
	{Trace-Ability, Inc.}
	{Developed algorithms for processing of optical signals from absorbance, luminescence and fluorescence measurements to extract quality control data suitable for FDA-compliant drug manufacturing. Wrote production code for automated peak finding on the chromatogramms, noise reduction, spectral normalization and outlier detection.  \\ \texttt{Signal processing}\slashsep\texttt{C\#}}
	\entry
	{2012 -- 2013}
	{UI and hardware for experimental drug manufacturing system}
	{UCLA}
	{Build an prototype liquid-handling system aided by radiation and optical sesnors. Devloped a C\#-based software that allowed for asynchronous operations of the hardware, UI and data aqusition boards. The system was featured in two academic pubilcations. \\ \texttt{Signal processing}\slashsep\texttt{WinForms}\slashsep\texttt{Async Programming}}
	\entry
	{2012 -- 2013}
	{Sensor signal processing for automated chemistry modules}
	{Siemens Healthcare}
	{Developed an algorithm for liquid transfers in a GMP-compliant automated drug manufacturing system. Processing a stream of data from various sensors, algoritm controlled liquid-handling hardware. The system was used in FDA-compliant manufacturing of radioapharmaceuticals.}
\end{entrylist}

%----------------------------------------------------------------------------------------
%	EDUCATION
%----------------------------------------------------------------------------------------

\cvsect{Education}

\begin{entrylist}
	\entry
	{2021 -- 2024}
	{Master in Data Science}
	{University of California Berkeley}
	{Coureses: ML at scale, Applied ML, Statistics for data sceience, Data engenering, Time series and panel data, Research design, Data Sceince programming}
	\entry
	{2002 -- 2005}
	{PhD in Chemistry}
	{Moscow State University}
	{Thesis: New aspects of Pd-catalyzed amination and its application in metallocene synthesis}
\end{entrylist}
%----------------------------------------------------------------------------------------
%	EXPERIENCE
%----------------------------------------------------------------------------------------
\cvsect{Experience}
\begin{entrylist}
	\entry
		{2022 -- Present}
		{Commercial Manager}
		{McMaster Nuclear Reactor, Canada}
		{Business developement and sales process for medical isotopes business}
	\entry
		{2015 -- 2019}
		{Chief Technology Officer}
		{Trace-Ability,Inc, Los Angeles, CA}
		{Technical founder, oversaw development and productization of software-hardware complex for quality control of radiopharmaceuticals.}
	\entry
		{2013 -- 2015}
		{Associate Project Sceintist}
		{University of Claifornia, Los Angeles}
		{Development of small molecules for imaging of inflammation in tumor models.}
\end{entrylist}

%----------------------------------------------------------------------------------------
%	ADDITIONAL INFORMATION
%----------------------------------------------------------------------------------------

%----------------------------------------------------------------------------------------

\end{document}
